% -*- root: ../mainThesis.tex -*-

\begin{abstract}
Well placement optimization is an important part of Petroleum Field Development. 
However, in order to improve the optimization procedures, it can be important to 
incorporate considerations like knowledge about the geology of the reservoir or 
about existing or planned well paths. This leads to additional constraints that 
have to be satisfied during the optimization. In this thesis we concentrate in 
particular on constraints on the well lengths and the distance between the wells.

We suggest an alternating projections method to deal with both constraints at the 
same time, and develop an efficient numerical method for the solution. Although we 
cannot prove that the method is convergent, numerical results of our implementation 
indicate that the approach works as intended.

An additional important contribution from this work is the 
implementation of a well index calculator.
% 
In reservoir simulation, the well index relates the flow rate 
and pressure of the wellbore to the pressure solution of the 
subsurface fluid flow system, and is therefore an essential 
part in computing resulting production volumes. 

We also implement an algorithm that, given a slanted 
well and the physical state of a reservoir, calculates 
the well indices for the well blocks that are intersected 
by the well. 
% 
In particular the well index calculation for deviated wells 
is a nontrivial task that is important for well placement 
optimization research.
% 
This task is already handled by some industry reservoir 
simulators, but the implementation is hidden from the 
end-user.

All of the implementations are meant to be an addition to FieldOpt, a petroleum 
field development optimization framework that is currently under development by 
the Petroleum Cybernetics Group at NTNU.
\end{abstract}

% Constrained well placement optimization is an 
% important part of Petroleum Field Development.
% %
% We formulate and solve the projections
% on a well length constraint and an inter-well
% distance constraint. We implement algorithms that
% efficiently solve the individual constraints.

% We then suggest an alternating projections method
% to deal with both constraints at the 
% same time, and although we cannot prove that the 
% method is convergent, numerical results of our 
% implementation imply that the approach works. 

% We also implement an algorithm that, given a well
% and the physical state of a reservoir, calculates
% the well indices for the well blocks that are 
% intersected by the well.

% All of the implementations are meant to be an
% addition to FieldOpt, a petroleum field development 
% optimization framework that is currently under
% development by the Petroleum Cybernetics Group at NTNU.