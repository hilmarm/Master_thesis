% -*- root: ../mainThesis.tex -*-

% #############################################
% DEFAULT TEXT
\chapter{Introduction}
%
This chapter gives the reader an overview of
our goals and the purpose of the work.
We wish to contribute to the work in petroleum field development 
by implementing constraint handling routines and a 
well index calculator as an aid to current well placement 
optimization methods. These implementation are made so that 
in the future they can be integrated in FieldOpt\cite{Fieldopt},
a petroleum field development optimization framework that will 
aid in the operations of producing hydrocarbons from the subsurface.
% 
The problem of placing wells is a substantial 
part of petroleum field development, and because of its 
importance we should use optimization procedures to augment 
the well placement decision-making. In this effort it is 
important that we have a way to measure the 
oil field production and its related costs, and that
we are able to define good constraints, in the sense 
that they are proper representations of petroleum engineering 
knowledge. After we have defined the task as a 
mathematical problem with associated constraints, we should 
try to develop and implement efficient methods to deal with
these constraints while maximizing income.
%
\paragraph{What is Petroleum Field Development(PFD).}
%
Petroleum Field Development is mainly concerned with maximizing 
the return of the financial
investment. We can gain financial revenue by increasing 
the recovery of oil from the reservoir or by reducing the
costs associated, e.g., drilling, labor, injection 
and the production of water.
%
%
\section{The general well placement problem}
%
\paragraph{What is the objective of well placement optimization.}
%
The problem of well placement optimization is the following:
Given some physical information about a reservoir we wish to place 
one or several wells in such a way that an objective is reached.
The objective is usually to maximize the net present value, which 
again is achieved by maintaining a high oil recovery while minimizing
costs at the same time.
%
This problem is one that has been studied extensively and many 
methods for optimizing the placement of the well have been proposed \cite{Bellout_paper}, 
both derivative based methods and derivative-free methods. 
%
%
\paragraph{How wells are parametrized.}
%
In the specific problem we study in this thesis,
$\textbf{x}$ represents the position of all wells
in the system, but generally 
it may also include other variables such as control
settings, bottom hole pressure or shape coefficients
in the case of curved wells.
%
A single straight well can be defined by the
coordinates of the heel and toe of the well.
%
As an example, four wells in three dimensional space
can be defined by a vector $\textbf{x}$ containing
%
$ N= N_{wells} \cdot N_{heel and toe} \cdot N_{dimension} = 4 \cdot 2 \cdot 3 =  24$
% 
real numbers, i.e., $\textbf{x} \in \mathbb{R}^{24}.$ \\
%
The general problem can be formulated as an unconstrained
optimization problem in the following way
%
\begin{equation}
\min_{\textbf{x} \in \mathbb{R}^N} J \left( \textbf{x} \right),
\label{introProb}
\end{equation}
%
where the objective function 
%
$J:\mathbb{R}^N \rightarrow \mathbb{R}$ 
%
maps a point
%
$\textbf{x} \in \mathbb{R}^N$ to a real number.
%
$J$ determines how well the objective is reached
in the point $\textbf{x}$. In a scenario of oil
production $J$ is typically be defined in such a 
way that oil production is maximized and the related costs
(e.g., the cost of drilling, well equipment etc.)
are minimized. 
%
%\mb{What are the main characteristics of the well 
%placement problem, e.g., model-based optimization, 
%what is a reservoir simulation, etc -- what makes 
%the problem special?} 
%
\paragraph{How is the objective computed, what is a
reservoir model/simulation.}
%
In order to evaluate the objective function in a point
$\textbf{x}$ a simulation is needed to determine the 
pressure distribution in the reservoir system. 
% 
This is typically done by providing a reservoir simulator, 
such as Eclipse \cite{Eclipse}, the physical state of 
the system which might include well block pressure,
permeability and well indices.
%
\paragraph{What is the well index.}
%
The well index relates wellbore flow rate and
pressure to well block quantities\cite{Wolfsteiner},
which is important for computing oil recovery.
%
The well index of a well block is uniquely determined 
by the well placement coordinates and is either left as 
a job to the reservoir simulator or computed and given 
as input by the user. 
% 
The reservoir simulator then computes the pressure 
distribution in the reservoir by numerically solving 
a set of partial differential equations (PDEs) which 
then implicitly determines the oil production rate.\\
%
The wells in the well placement problem are treated
as straight line segments or as continuous chains of
straight line segments inside a reservoir of blocks.
%
The blocks of the reservoir have six planar faces and 
every face of a block is either shared with the face of 
another block or lies on the boundary of the reservoir 
domain.
%
%\mb{Extension of the well placement problem,
%e.g., we could optimize for well completions, 
%well controls}
%
It is possible to extend the model to not 
only consider the placement of wells. One
could also include things such as time dependent 
well control variables.
%
This would result in a more complex variation of 
the original problem (\ref{introProb}), where oil 
production is treated over a longer time span 
instead of being instantaneous. 
%
Including the new variables
could look like the following:
%
\begin{equation}
\min_{\substack{\textbf{x} \in \mathbb{R}^N\\\textbf{y} \in \mathbb{R}^k }} \sum_{t} J \left( \textbf{x},\textbf{y}, t \right),
\end{equation}
where $\textbf{x}$ are the well placement coordinates,
$\textbf{y}$ are the well control variables and $t$ is
a time variable.
%
\paragraph{How we solve for the well placement
problem.}
%
Gradients of the objective function with respect to well placement variables 
are not readily available and are likely to be be discontinuous. As noted by Bellout et al.
\cite{Bellout_paper} the well placement problem does not appear to be as amenable using gradient-based methods 
because these approaches can get trapped in local minima.
An alternative is to treat it using derivative-free approaches
such as genetic algorithms, stochastic perturbation methods and particle swarm optimization.
%
%\mb{Why are constraints important for the 
%well placement problem, what type of constraints
%are reasonable to include in the problem}
\paragraph{Well placement constraints.}
% 
It is important that the search for well configurations
is constrained by realistic petroleum engineering 
considerations for how best to develop the field, 
e.g., knowledge about the geology and flow properties
of the reservoir and information about existing or
planned well paths and facilities.
% 
It is crucial for an efficient well placement optimization
effort to articulate this type of information into a
properly defined objective function with constraints
that can be treated using mathematical programming.
%
In order for our current well placement model 
to be practically useful there are several 
limitations to the placement of wells. 
%
These include, but are not restricted to, constraints 
on the length of a well (well length constraint), 
how close two wells can be to each other (inter-well 
distance constraint) and where a well is allowed to 
reside (well domain constraint).
%
All of these restrictions on the wells result in 
a number of constraints on the well coordinates.\\
%
%\mb{Important: How do this constraints look like, 
%why are they easy/difficult to deal with? How do 
%we describe them?}
%
\begin{itemize}
	\item Well length restrictions require that wells should not be too short
	but also not too long. We require that the heel $\textbf{x}^h$ and toe
	$\textbf{x}^t$ of every well should be separated by at least a distance
	$L_{\min}$ but not more than $L_{\max}$.
%%%%%%%%%%%%%%%
	\item The restriction on how close wells can be to each other results in the inter-well distance
	constraint. Essentially wells either interfere if they are placed too close to each other
	or it makes drilling either dangerous or impossible to perform. For all pairs of wells we require
	that every point of one well is at least a distance $d$ away from every point of the other
	well.
%
	\item A well location restriction gives a domain constraint, which demands that
	a given well position be in some predefined feasible domain $\Omega_{wd}$.
	Although this constraint looks simple enough, a domain
	$\Omega_{wd}$ may be arbitrarily defined and needs not be simple or convex, making 
	projections on it difficult to find.
\end{itemize}
%
%
\paragraph{How do we solve for the constrained well placement
problem.}
%
This thesis aims to contribute to the well placement problem
by taking well coordinates from a single unconstrained optimization
step and developing a way to project coordinates that violate
constraints back into feasible space in such way that the 
coordinates of the wells are moved as little as possible.
This process is then continued iteratively so that after every
unconstrained optimization step we project wells so that all
constraints are satisfied. In addition an algorithm to calculate
the well index for deviated wells (i.e., not strictly vertical)
described by Shu in \cite{Shu_Paper} is implemented. This
calculation is currently not handled by FieldOpt itself but by 
the reservoir simulators.
%
\paragraph{Tasks for this thesis}
%
\begin{itemize}
 	\item {Determine and implement constraints 
 	that are physically reasonable.}
 	\item {Implement a routine to deal with the 
 	constraint handling as an optimization problem. 
 	The routine projects non-feasible coordinates 
 	onto a feasible space.}
 	\item {Implement well index calculation for 
 	deviated (i.e., not perfectly horizontal or 
 	vertical) wells.}
 \end{itemize} 
%
% 
In Chapter 2 we will introduce the well placement problem and formulate the
well constraints in a more detailed manner, in addition we will explain
some ideas for how to handle multiple constraints. In Chapter 3 we will solve
the individual constraint problems and in Chapter 4 a method for computing 
the well index for blocks is outlined.
%




