% -*- root: ../mainThesis.tex -*-
\chapter{Summary and discussion}
%
We summarize the results of the projection methods
and the well index calculator and comment on the
main points of them. 

The main goal of the thesis was the solution of two sub-problems 
occurring during the tackling of well-placement optimization, namely 
handling of well constraints and well index calculations. These
are important because they reflect physical properties of the 
reservoir and are essential for a reliable and practical way
of solving the overall well placement problem. 
%
\section{Projection to feasible space}
%
The first problem we discussed was the handling of 
well-placement constraints. Here we introduced and 
implemented a method based on alternating projections, 
where one of the projections could be solved analytically, 
while the other projection had to be solved numerically.

The alternating projection method for satisfaction
of inter-well distance and well distance constraint
was implemented. The well distance constraint was
solved and the analytical solutions were derived
and implemented. 

The main issue was the derivation
of an accurate method for the projection on the 
inter-well distance constraint. The idea of splitting
up the problem into $k$-point solutions leads to
different cases. The main sub-case is finding the
roots of the sixth degree polynomial in \eqref{eq:six_degree_poly},
which can be done efficiently with arbitrarily high
precision. The implemented version preformed well
overall and managed to even solve cases where one
would expect numerical issues, such as when
all wells are along the same line. However, the 
implementation was not able to solve one case 
with parallel line segments which is presented 
in Chapter 8. The reason for this is unknown.
%
\section{Well index calculation}
%
The goal for the well index calculator was to
implement an alternative method capable of 
dealing with deviated and slanted wells.
%%%%%%%%%%%%%%%%%%
The method works by taking weighted means of the well 
indices one would obtain for centered wells parallel to 
the block axes, with weights depending on the projections 
of the well on these axes. To do this we require  
computations of entry and exit points of each of the
blocks penetrated by the well.

The implemented algorithm for finding well
indices compared well to industry standards
for well index calculations.
%
