% -*- root: ../mainThesis.tex -*-
%
\chapter{Implementation}
%
Here we briefly explain the implementations that were
made and the most important part of the code can
be seen in the Appendix. For the full code we refer
to the author's Github repository\cite{Hilmar_git}.
%
\section{Software}
All code was written in Qt 5.5, a cross-platform application and UI 
framework for C++ developed by The Qt Company\cite{QtCompany}.
%
The code makes great use of the Eigen library\cite{Eigen_library} 
which is a template library for linear algebra that includes
matrix and vector classes and algorithms for matrix decompositions.
This is required in the implementation of the inter-well distance
projection for finding the eigenvalue decomposition needed in
\eqref{eq:interwell_s}. The vector class \texttt{Vector3d} of
the Eigen library is also used frequently throughout most
implementations as it contains many useful functions.
%
In order to find the roots of polynomials we make use of the 
RPOLY library \cite{Rpoly}, which is an implementation of the 
Jenkins-Traub algorithm\cite{Jenkins_traub}. This is needed for
solving the sixth degree equation \eqref{eq:six_degree_poly} 
which is the most important part of the inter-well distance 
projection.
%
%
\section{Well length constraint projection}
%
The function \texttt{well\_length\_projection\_eigen()}, takes
the initial coordinates of the heel and toe of a well, and by
calculating the distance between them it determines which of the 
solutions \eqref{eq:solution_min_triv} -- \eqref{eq:wl_formula_last}
it should return.  
%
%
\section{Inter-well distance constraint projection}
%
In the function \texttt{interwell\_constraint\_projection\_eigen()}
which was implemented we take the initial coordinates of two
wells and a distance $d$ as input. Then we try to find solutions by 
moving as few points as possible. First we try to move only two
points by using \texttt{well\_length\_projection\_eigen()}.
If some two-point solution is feasible for the complete
problem we return the best one. If no two-point solution is
found we try the three-point solutions. This is done by
building $A$ and $b$ according to \eqref{eq:A_b_three_points},
and then running \texttt{kkt\_eq\_solutions\_eigen($A,b$)} which
returns all candidate solutions of \eqref{eq:interwell_matrix}.
If one or more feasible solutions are found we pick the best one.
If no solution has been found yet we must have a four-point
solution. We build $A$ and $b$ according to \eqref{eq:A_b_three_points}
and pick the best solution returned by 
\texttt{kkt\_eq\_solutions\_eigen($A,b$)}.
%
\section{Alternating projections}
%
Now that both well length projection and inter-well distance
projection are available, the method of alternating projections
is simply done by running one projection after the other inside
a while loop until the well positions are feasible. Note that
we can also change the order of the projections, which might
impact the solution. In our implementation the inter-well
distance projection was done first.
%
\section{Well index calculation and intersecting blocks}
%
In order to compute the well indices for the well blocks
of a reservoir, we first need to determine which blocks
are penetrated by a well and what the entry and exit points
are. If these two steps are handled, then computing the well
index for each block is done by computing the well block
projections as shown in Figure \ref{fig:well_index} and then
supplying the block dimensions and permeabilities and using
\eqref{eq:well_index}. Here we describe the algorithm
used to find the well blocks which are intersected by a well
and in which points the intersections occur. Well blocks will
only be referred to as blocks.
%
Assume that \texttt{GetblockEnvelopingPoint($p$)} returns a block 
that contains the point $p$. Assume also that \texttt{FindIntersectionPoints}(block, line)
calculates the two intersection points 
between a line segment and a block.\\
%
A list of intersected blocks and their entry and exit points are created
and returned at the end of the algorithm.
%
\begin{algorithm}
\caption{Input: reservoir(blocks), line(start point, end point) }\label{alg:block_intersection}
\begin{algorithmic}[1]
\Procedure{Find intersected blocks}{}
\State first block $\gets$ \texttt{GetblockEnvelopingPoint}(start point)
\State last block $\gets$ \texttt{GetblockEnvelopingPoint}(end point)
\State Set current block = first block
\State 
\While {current block $!=$ last block}
	\State intersection points $\gets$ \texttt{FindIntersectionPoints}(current block, line)
	\State add intersection points to list
	\State add current block to list
	\State
	\State new point $\gets$ Move small distance out of current block in direction of \hspace{10mm} .\hspace{10mm} end point
	\State 
	\State current block $\gets$ \texttt{GetblockEnvelopingPoint}(new point)
	\State 
\EndWhile

\State Add last block to list
\State Return lists

\EndProcedure
\end{algorithmic}
\end{algorithm}
