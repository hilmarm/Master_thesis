% -*- root: ../mainThesis.tex -*-
%
\chapter{Problem formulation}
%
This chapter
introduces the overall well placement optimization problem
and how we represent wells. Moreover, we introduce the different 
constraints more clearly and indicate how we intend to handle 
several constraints at the same time.
%
For the rest of this thesis we will assume that a well connects
its heel and toe in a straight line.
We describe well $i$ with the coordinates of its heel and toe
($\textbf{x}_i^h, \textbf{x}_i^t$) $\in \mathbb{R}^3 \times \mathbb{R}^3$.
If we have multiple wells we collect all $N$ wells in a single variable
$\textbf{x} \in (\mathbb{R}^3 \times \mathbb{R}^3)^N$.
%
\section{Well problem formulation}
We define the overall well placement optimization problem as
%
\begin{equation}
\begin{aligned}
\min_{\textbf{x} \in (\mathbb{R}^3 \times \mathbb{R}^3)^N} J\left( \textbf{x} \right) \\
\text{such that} \hspace{5px} \textbf{x} \in \Omega,
\label{problemOmega}
\end{aligned}
\end{equation}
% 
% [p.16 B.Grimstad, PhD thesis]
% [p.22 ++ O.Isebor, Phd Thesis]
% 
where $J: (\mathbb{R}^3 \times \mathbb{R}^3)^N \rightarrow \mathbb{R} $ is a
user-defined objective function that maps the current well positions to a real
number. The choice of $J$ should maximize oil production or the net present value
while minimizing various costs such as well drilling costs, well length costs and
other factors. For now the exact definition of $J$ is left open, as we concentrate
on satisfying the constraints.

The domain $\Omega$ is the set of all well coordinates $\textbf{x}$ that satisfy a 
set of linear and nonlinear constraints that enforce certain restrictions which we 
shall define below.
%
We need to select, develop and implement constraints with the
physical objective of drilling in mind. 
%
This means constraint types for the positioning of wells
in a reservoir should be reasonable representations of 
engineering restrictions and priorities for how a petroleum 
field should be developed.
% 
To restrain overall well configuration in field development,
in this work we define three types of constraints: a well 
length, an inter-well distance and a reservoir boundary 
constraint.
%
\subsection{Well length constraint}
First we must define a metric or distance function
$g:\mathbb{R}^3 \times \mathbb{R}^3 \rightarrow \mathbb{R}$ 
which takes two points in three dimensional space and maps 
them to a real number. We use the most natural choice, namely
the Euclidean distance. For two points
$\textbf{p}, \textbf{q} \in \mathbb{R}^3$ the distance between
them is
%
\begin{equation}
 g(\textbf{p},\textbf{q}) := \| \textbf{p}-\textbf{q} \| = \sqrt{\sum_{i=1}^3 (q_i-p_i)^2}.
 \end{equation}

A well should not be longer than $L_{\max}$ nor shorter than $L_{\min}$, 
or equivalently, the length should be in the interval $[L_{\min},L_{\max}]$.
Note that we will also require that $L_{\min} > 0$, i.e., we don't allow
wells of zero length.  
The well length constraint can now be formulated as
%
\begin{align}
\| \textbf{x}_i^h - \textbf{x}_i^t \| \leq L_{\max}, \\
\| \textbf{x}_i^h - \textbf{x}_i^t \| \geq L_{\min}, \\
\text{for all} \quad i = 1,2,\dots,N. 
\end{align}

% =============================================
\subsection{Inter-well distance constraint}
Every pair of wells should be at least some minimum distance $d$ apart.
This means that every single point of one well needs to be at least a
distance $d$ from all points of every other well.
If a well is the straight line between the heel and toe of the well then
this is equivalent to requiring that
%
\begin{align}
\| (\textbf{x}_i^h + \lambda_1 (\textbf{x}_i^t - \textbf{x}_i^h  )) - (\textbf{x}_j^h + \lambda_2 (\textbf{x}_j^t - \textbf{x}_j^h)) \| \geq d,\\
\quad \lambda_1, \lambda_2 \in [0,1], \\
\text{for all pairs } (i,j) \text{ of wells with } i \neq j.
\end{align}
%
%
\subsection{Reservoir bound constraint}
%
The reservoir is made up of grid blocks which are convex polyhedra,
but the reservoir itself is not necessarily convex. All 
wells can be required to lie in a feasible domain, and heel and 
toe might have different feasible domains. Domain bounds should
reflect the geological situation and it might also be natural to
assume some restriction on heel position that is given by the 
drilling operator. If all wells lie in a feasible domain
we say that $\textbf{x}$ is feasible, or simply that $\textbf{x} \in \Omega_{wd}$.
Due to lack of time we were not able to define and solve a reservoir
bound constraint, so for the rest of the thesis we will assume that
all possible positions $\textbf{x}$ satisfy the reservoir bound constraint.\\

%
Collecting all constraints we can rewrite equation \eqref{problemOmega} as
%
\begin{align}
\min_{\textbf{x}} \hspace{0.6mm} &J\left(\textbf{x}\right) \\
\intertext{such that}
\| \textbf{x}_i^h - \textbf{x}_i^t \| &\leq L_{\max}, \\
\| \textbf{x}_i^h - \textbf{x}_i^t \| &\geq L_{\min},\label{eq:min_non_convex} \\
\intertext{for all $\quad i = 1,2,\dots,N,$} 
\| (\textbf{x}_i^h + \lambda_1 (\textbf{x}_i^t - \textbf{x}_i^h  )) - (\textbf{x}_j^h + \lambda_2 (\textbf{x}_j^t - \textbf{x}_j^h)) \| &\geq d,\label{eq:dis_non_convex}\\
\quad \lambda_1, \lambda_2 &\in [0,1], \\
\intertext{for all pairs $(i,j) \text{ of wells with } i \neq j,.$} \nonumber
\end{align}
%
\section{Projection of multiple constraints}
%
Given a set of well coordinates $\textbf{x}_k$ and an
objective function $J$, an unconstrained optimization 
step, $O$, is performed in order to achieve an improved position
\begin{equation}
O: \textbf{x}_k \mapsto  \tilde{\textbf{x}}_{k+1}.
\end{equation}
%
If a position $\textbf{x}$ satisfies all constraints we say
that $\textbf{x} \in \Omega$.
%
If the improved position does
not satisfy all constraints then the coordinates need to be
projected back to feasible space by some projection method $\mathcal{P}$.
Ideally we want to find a projection method $\mathcal{P}$
%
\begin{equation}
\mathcal{P}: \tilde{\textbf{x}}_{k+1} \mapsto \textbf{x}_{k+1},
\end{equation}
%
that solves the problem
%
\begin{align}
\min_{\textbf{x}_{k+1}} \| \tilde{\textbf{x}}_{k+1} - \textbf{x}_{k+1} \|, \label{eq:proj_opt}\\
\text{such that} \quad \textbf{x}_{k+1} \in \Omega,
\end{align}
%
i.e., a projection that moves a position back into feasible
space by moving the points as little as possible.
%
%
\section{Simultaneous constraint projection}
% 
Even if an analytical solution to \eqref{eq:proj_opt} exists, i.e., 
solving the well length constraint and inter-well distance constraint at
the same time, it is probably very difficult because both
\eqref{eq:min_non_convex} and \eqref{eq:dis_non_convex} are non-convex.

Using numerical solvers for constrained optimization, such as \texttt{fmincon()}
in MATLAB\cite{Matlab}, is problematic because of the implementation of 
the constraints. Especially the well distance constraint is pretty difficult 
to implement because of the piecewise definition of the closest points.
Therefore we need to simplify our approach and look for a possibly (and 
probably) suboptimal solution of \eqref{eq:proj_opt} if we wish find a working
projection. 
%
\subsection{Method of alternating projections}
%
Alternating projections is a standard approach for this kind of problem.
If we know how to project onto the two sets $C$ and $D$ with the 
projections $\mathcal{P}_C$ and $\mathcal{P}_D$ respectively, then the
alternating projection method is defined as
%
\begin{align}
x_{k+1} = \mathcal{P}_C \left( \mathcal{P}_D ( x_k ) \right)
\label{eq:alt_proj_sec}
\end{align}
%
Moreover, if the sets $C$ and $D$ are convex and their intersection is non-empty, then the 
sequence \eqref{eq:alt_proj_sec} will converge to some point in this intersection.

Although we have no idea how to compute the whole projection analytically, 
we can still compute the individual projections onto the feasible sets for 
the well length constraint ($\mathcal{P}_{l}$), and the projection for the inter-well 
distance constraint for two wells ($\mathcal{P}_{d}$). The details will be
discussed in Chapter 3.

Thus we can attempt to use
the method of alternating projections to find a feasible point.
%
Neither the feasible points for the well length constraints 
nor the feasible points for the inter-well distance constraint 
form convex sets, so we cannot guarantee any global convergence of
the method.

However, in a result\cite{Lewis_Luke_Malick} by Lewis, Luke and Malick,
\cite[Thm. 5.16]{Lewis_Luke_Malick} states the following: If we have two sets,
$A$ and $B$, with $A$ super-regular (see \cite[Def. 4.3]{Lewis_Luke_Malick}) 
and $B$ closed, and with non-opposing normal vectors to the sets at every point
in their intersection, it then follows that the alternating projection converges
locally R-linear to a point in $A \cup B$.
%
From \cite[Proposition 4.8]{Lewis_Luke_Malick} we have that amenability implies super regularity, and
the remark one line earlier states that if $A$ is defined by $C^1$ inequality 
constraints and the Mangasarian-Fromowitz constraint qualification \cite[Def. 12.6]{Nocedal_Wright} 
(or the stronger linear independence constraint qualification) holds, then A is amenable.
%
By calculating the gradient of the well length constraint \eqref{eq:min_non_convex}
we get that
%
\begin{align}
\nabla \left( \frac{1}{2} \| x - y \|^2 - \frac{1}{2}L_{\min}^2 \right) = \begin{bmatrix} x-y \\y-x \end{bmatrix},
\end{align}
%
which is non-zero for all $x \neq y$. Now since we require that wells have non-zero
length, this implies that $x \neq y$. This means that the well length constriction
satisfies the linear independence constraint condition which in turn implies that
the set of feasible points for the well length constraint is super-regular.
%
Therefore it follows that the alternating projection of the well length constraint
and the inter-well distance constraint is locally convergent. Note, however, that 
this result for the projections $\mathcal{P}_{l}$ and $\mathcal{P}_{d}$ only
holds for two wells.
%
\subsection{Inter-well distance projection on more than two wells}
%
The treatment of the inter-well distance constraint is particularly
difficult for multiple wells because it puts requirements on every 
pair of wells, and thus it increases in complexity as the number of 
wells increases.
%
In order to solve the inter-well distance constraint problem in a
system with more than two wells, we apply the projection $\mathcal{P}_{d}$
to two wells at a time and hope that it eventually converges. Call this
projection process $\mathcal{P}_{m}$.
%
\subsection{Alternating projection pseudo code}
%
Here we provide a pseudo code of the algorithm for the
locally convergent alternating projection method for two
wells. Note that by replacing $\mathcal{P}_{d}$ with
$\mathcal{P}_{m}$ the code can also attempt to solve 
the projection problem for more than two wells.
%

%
\begin{algorithm}
\caption{Constraint handling}\label{alg:constraint_handling}
\begin{algorithmic}[1]
\Procedure{Project coordinates to feasible space}{}
	\State Get initial coordinates $\textbf{y} \gets \textbf{x}$ 
	\State
	\While{$\textbf{x}_{k}$ not feasible}
		\State{$\textbf{y} \gets \mathcal{P}_{l}(\textbf{y})$}
		\State{$\textbf{y} \gets \mathcal{P}_{d}(\textbf{y})$}
	\EndWhile
	\State Return $\textbf{y}$
\EndProcedure
\end{algorithmic}
\end{algorithm}
%
%
The complete iterative optimization process can be done by using
the optimization step $O$ and the projections $\mathcal{P}_{l}$ and
$\mathcal{P}_{d}$. Note again that the process extends to handling
multiple wells if we replace $\mathcal{P}_{d}$ with $\mathcal{P}_{m}$.
%
\begin{algorithm}
\caption{Iterative optimization method}\label{alg:iterative_optimization}
\begin{algorithmic}[1]
\Procedure{Optimize constrained problem}{}
	\State Get initial coordinates $\textbf{x}_{0}$
	\State
	\State $ k = 0 $
	\While{$J(\textbf{x}_{k}$) not optimal}
		\State{$\tilde{\textbf{x}}_{k+1} \gets O(\textbf{x}_{k})$}
		\While{$\tilde{\textbf{x}}_{k+1}$ not feasible}
			\State{$\tilde{\textbf{x}}_{k+1} \gets \mathcal{P}_{l}(\tilde{\textbf{x}}_{k+1})$}
			\State{$\tilde{\textbf{x}}_{k+1} \gets \mathcal{P}_{d}(\tilde{\textbf{x}}_{k+1})$}
		\EndWhile
	\State $\textbf{x}_{k+1} \gets \tilde{\textbf{x}}_{k+1}$
	\State $ k = k+1 $
	\EndWhile
	\State Return $\textbf{x}_{k}$
\EndProcedure
\end{algorithmic}
\end{algorithm}
%

The solution for each individual projection will be handled in the next chapter.
