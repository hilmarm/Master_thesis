% -*- root: ../mainThesis.tex -*-

% #############################################
% DEFAULT TEXT
\chapter{Background}
%
Some general background of methods and mathematics which will be used in this paper
%
\section{Method of Lagrange multipliers}
%
In the area of mathematical optimization theory the method of Lagrange multipliers is 
a way of finding local minima and maxima of a function subject to one or more equality 
constraints. E.g. solve:
%
\begin{equation}
\min_{\textbf{x}} f(\textbf{x}) \quad \text{subject to} \quad  c_i(\textbf{x}) = 0
\label{lagrangeEx}
\end{equation}
%
Where $\textbf{x} \in \mathbb{R}^n $ and $f,c_i : \mathbb{R}^n \rightarrow \mathbb{R} $.
By introducing the \textit{Lagrangian function} (\cite{Nocedal_Wright})
%
\begin{equation}
\mathcal{L} (\textbf{x}, \lambda) = f(\textbf{x}) - \lambda c_1(\textbf{x}),
\label{lagrangian}
\end{equation}
%
where $\lambda \in \mathbb{R}$ is a Lagrange multiplier. It can be shown that for every 
local minimum $\textbf{x}^*$ there exists a scalar $\lambda^*$ such that:
%
\begin{equation}
\nabla_{\textbf{x}, \lambda} \mathcal{L} (\textbf{x}^*, \lambda^*) = 
\begin{bmatrix}
	0 \\
	0 \\
	\vdots \\
    0   
\end{bmatrix}
\label{optimumNecessary}
\end{equation}
%
